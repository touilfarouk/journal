\section*{Sommaire}\label{sec:sommaire}
\hypertarget{page:sommaire}{}
Voici un tableau récapitulatif du contenu du livre :

\begin{tabularx}{\textwidth}{@{}Xr@{}} % X = flexible column, r = right aligned
\textbf{Section} & \textbf{Page} \\
\addlinespace[0.5ex]

\multicolumn{2}{@{}l}{\textbf{Partie I : Les bases de Jetpack Compose}} \\[0.5ex]
\hyperref[sec:intro]{Introduction : pourquoi Compose change tout} \dotfill & 1 \\
\hyperref[sec:xml]{Dire adieu au XML et adopter le déclaratif} \dotfill & 3 \\
\hyperref[sec:composable]{Comprendre la logique des @Composable} \dotfill & 6 \\
\hyperref[sec:cycle]{Cycle de vie d’un composable et rôle du compilateur} \dotfill & 9 \\[1ex]

\multicolumn{2}{@{}l}{\textbf{Partie II : Construire une vraie application}} \\[0.5ex]
\hyperref[sec:projet]{Notre projet fil rouge : l’application To-Do} \dotfill & 13 \\
\hyperref[sec:activity]{Une seule Activity, zéro Fragment : la nouvelle approche} \dotfill & 16 \\
\hyperref[sec:room]{Stocker nos données avec Room et SQL} \dotfill & 20 \\
\hyperref[sec:navigation]{Naviguer entre les écrans avec Compose Navigation} \dotfill & 24 \\
\hyperref[sec:viewmodel]{Gérer l’état avec ViewModel} \dotfill & 28 \\
\hyperref[sec:datastore]{Enregistrer les préférences avec DataStore} \dotfill & 32 \\[1ex]

\multicolumn{2}{@{}l}{\textbf{Partie III : Rendre l’expérience moderne et fluide}} \\[0.5ex]
\hyperref[sec:ui]{Créer des composants UI personnalisés} \dotfill & 37 \\
\hyperref[sec:material]{Utiliser Material Design et Scaffold} \dotfill & 41 \\
\hyperref[sec:splash]{Ajouter un splash screen animé} \dotfill & 45 \\
\hyperref[sec:swipe]{Implémenter Swipe-to-delete et Undo} \dotfill & 49 \\
\hyperref[sec:animations]{Ajouter des transitions et animations} \dotfill & 53 \\[1ex]

\multicolumn{2}{@{}l}{\textbf{Partie IV : Développement avancé et bonnes pratiques}} \\[0.5ex]
\hyperref[sec:hilt]{Injection de dépendances avec Hilt} \dotfill & 58 \\
\hyperref[sec:darkmode]{Support du mode sombre et gestion des thèmes} \dotfill & 62 \\
\hyperref[sec:archi]{Architecture et organisation du projet Compose} \dotfill & 66 \\
\hyperref[sec:pièges]{Pièges courants et comment les éviter} \dotfill & 70 \\
\hyperref[sec:futur]{Vers l’avenir : Compose et l’écosystème Android} \dotfill & 74 \\

\end{tabularx}
