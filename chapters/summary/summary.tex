\section*{Sommaire}
\hypertarget{page:sommaire}{}  % <- la cible exacte du début du sommaire


{\footnotesize \textcolor{darkgray}{La navigation dans le sommaire et les chapitres est bidirectionnelle. }} 
{\footnotesize \textcolor{gray}{\textit{Chaque titre de section est cliquable et renvoie à la page correspondante. De même, chaque page de chapitre comporte un lien de retour vers le sommaire pour une navigation aisée.}}}

\begin{tabularx}{\textwidth}{@{}Xr@{}} % X = flexible column, r = right aligned
\textbf{Section} & \textbf{Page} \\
\addlinespace[0.5ex]

\multicolumn{2}{@{}l}{\footnotesize \textcolor{darkgray}{{Partie I : Les bases de Jetpack Compose}}} \\[0.5ex]
\hyperref[sec:intro]{Introduction : pourquoi Compose change tout} \dotfill & {\footnotesize \textcolor{darkgray}{{1}}} \\
\hyperref[sec:chapter-1]{Créer notre premier projet Compose} \dotfill & {\footnotesize \textcolor{darkgray}{{4}}}\\


\end{tabularx}
