% Section avec titre coloré et cliquable vers le sommaire
\section[\texorpdfstring{Créer notre premier projet Compose}{Créer notre premier projet Compose}]%
{\hyperref[sec:sommaire]{\textcolor{orange}{Créer notre premier projet Compose}}}%
\label{sec:chapter-1}

{\footnotesize
\textcolor{darkgray} {Avant de créer notre premier projet Jetpack Compose, il est nécessaire de disposer
d’un environnement de développement prêt à l’emploi.  
Vous pouvez télécharger et installer Android Studio gratuitement depuis le site officiel : }
\href{https://developer.android.com/studio}{\textcolor{blue}{Télécharger Android Studio}}.
}
\vspace{1.5em}


% Texte descriptif des étapes (pas de lignes vides dans \textcolor)
{\footnotesize
\textcolor{darkgray}{\textbf{Première étape :} Lors de la création d’un nouveau projet sous Android Studio,
il faut commencer par choisir \textit{Empty Activity}, ce qui permet de démarrer
avec une structure minimale et totalement personnalisable.}
}

\vspace{0.5em}

{\footnotesize
\textcolor{darkgray}{\textbf{Deuxième étape :} Ensuite, il est nécessaire de donner un nom à votre projet,
définir le langage (Kotlin de préférence) et choisir l’emplacement de sauvegarde. \textcolor{darkgray}{\textbf{Zoomer sur les deux photo pour voir les détails.}}} 
}

% Custom command: \mygraphics[<options>]{<image-path>}{<title>}
\newcommand{\mygraphics}[3][]{%
  \tcbox[
    enhanced,
    capture=minipage,
    boxsep=0pt,top=0pt,bottom=0pt,left=0pt,right=0pt,
    boxrule=0.4pt,
    drop fuzzy shadow,
    nobeforeafter,
    colback=black!75!white,
    toptitle=2pt,bottomtitle=2pt,
    center title,
    fonttitle=\small\sffamily,
    title={#3},
    width=(\linewidth-4mm)/2,
    height=6cm,
    colbacktitle={black},
    watermark zoom=1.0,
    watermark graphics={#2}
  ]{}%
}

% Images côte-à-côte
\noindent
\mygraphics{images/chapter-1/android-studio-1.png}{Première étape : Choisir Empty Activity}%
\hfill
\mygraphics{images/chapter-1/android-studio-2.png}{Deuxième étape : Nommer le projet}

\vspace{1em}

Troixième étape : Ouvrir MainActivity.kt
\begin{ktlst}

class MainActivity : ComponentActivity() {
    override fun onCreate(savedInstanceState: Bundle?) {
        super.onCreate(savedInstanceState)
        enableEdgeToEdge()
        setContent {
            GestiondesTachesTheme {
                Scaffold(modifier = Modifier.fillMaxSize()) { innerPadding ->
                    Greeting(
                        name = "Android",
                        modifier = Modifier.padding(innerPadding)
                    )
                }
            }
        }
    }
}



\end{ktlst}
\vspace{1em}
Je vais vous guide a ecrire du code sur le fichier MainActivity.kt, pour voir comment on peut utiliser jetpack compose afin de creer un compteur simple en utilisant le conpodent Boutton et Text,
le boutton pour generer des nombre sur le click, et le chemp texte pour aficher les nombres .
pour cela effacer la fonction Greeting et ecrire le code suivant: 






Documentation Officielle (Hilt) : https://dagger.dev/hilt/

Documentation Officielle : https://developer.android.com/topic/libraries/architecture/datastore

Documentation Officielle : https://developer.android.com/jetpack/androidx/releases/room

de ma part et mon propre style de code, je commance par definire les versions des librairies dan le fichier libs.versions.toml





 
\section{\textcolor{orange}{Déclaration d'une classe en Kotlin}}\label{sec:classes}
\begin{ktlst}

% [versions]
hilt = "2.52"
ksp = "2.0.21-1.0.25"
roomRuntime = "2.7.2"
datastorePreferences = "1.1.7"

% [libraries]

androidx-datastore-preferences = { module = "androidx.datastore:datastore-preferences", version.ref = "datastorePreferences" }
androidx-room-compiler = { module = "androidx.room:room-compiler", version.ref = "roomRuntime" }
androidx-room-ktx = { module = "androidx.room:room-ktx", version.ref = "roomRuntime" }
androidx-room-runtime = { module = "androidx.room:room-runtime", version.ref = "roomRuntime" }
hilt-android = { group = "com.google.dagger", name = "hilt-android", version.ref = "hilt" }
hilt-compiler = { group = "com.google.dagger", name = "hilt-compiler", version.ref = "hilt" }
hilt-navigation-compose = { group = "androidx.hilt", name = "hilt-navigation-compose", version = "1.2.0" }

% [plugins]
hilt-android = { id = "com.google.dagger.hilt.android", version.ref = "hilt" }
ksp = { id = "com.google.devtools.ksp", version.ref = "ksp" }

\end{ktlst}


\begin{ktlst}
plugins {
    alias(libs.plugins.android.application) apply false
    alias(libs.plugins.kotlin.android) apply false
    alias(libs.plugins.kotlin.compose) apply false

    alias(libs.plugins.hilt.android) apply false   
    alias(libs.plugins.ksp) apply false          

}
\end{ktlst}

\begin{ktlst}
plugins {
    alias(libs.plugins.android.application)
    alias(libs.plugins.kotlin.android)
    alias(libs.plugins.kotlin.compose)
    //  apply Hilt + KSP only here
    alias(libs.plugins.hilt.android)
    alias(libs.plugins.ksp)

}
\end{ktlst}

\begin{ktlst}
dependencies {

    // Hilt dependencies
    implementation(libs.hilt.android)
    ksp(libs.hilt.compiler)
    implementation(libs.hilt.navigation.compose)

    // Room components
    implementation(libs.androidx.room.runtime)
    ksp (libs.androidx.room.compiler)
    implementation(libs.androidx.room.ktx)

    // DataStore Preferences
    implementation(libs.androidx.datastore.preferences)
}
\end{ktlst}










