% Section avec titre coloré et cliquable vers le sommaire
\section[\texorpdfstring{Créer notre premier projet Compose}{Créer notre premier projet Compose}]%
{\hyperref[sec:sommaire]{\textcolor{orange}{Créer notre premier projet Compose}}}%
\label{sec:chapter-1}

{\footnotesize
\textcolor{darkgray} {Avant de créer notre premier projet Jetpack Compose, il est nécessaire de disposer
d'un environnement de développement prêt à l'emploi.  
Vous pouvez télécharger et installer Android Studio gratuitement depuis le site officiel : }
\href{https://developer.android.com/studio}{\textcolor{blue}{Télécharger Android Studio}}.
}
\vspace{1.5em}


% Texte descriptif des étapes (pas de lignes vides dans \textcolor)
{\footnotesize
\textcolor{darkgray}{\textbf{Première étape :} Lors de la création d'un nouveau projet sous Android Studio,
il faut commencer par choisir \textit{Empty Activity}, ce qui permet de démarrer
avec une structure minimale et totalement personnalisable.}
}

\vspace{0.5em}

{\footnotesize
\textcolor{darkgray}{\textbf{Deuxième étape :} Ensuite, il est nécessaire de donner un nom à votre projet,
définir le langage (Kotlin de préférence) et choisir l'emplacement de sauvegarde. \textcolor{darkgray}{\textbf{Zoomer sur les deux photo pour voir les détails.}}} 
}

% Custom command: \mygraphics[<options>]{<image-path>}{<title>}
\newcommand{\mygraphics}[3][]{%
  \tcbox[
    enhanced,
    capture=minipage,
    boxsep=0pt,top=0pt,bottom=0pt,left=0pt,right=0pt,
    boxrule=0.4pt,
    drop fuzzy shadow,
    nobeforeafter,
    colback=black!75!white,
    toptitle=2pt,bottomtitle=2pt,
    center title,
    fonttitle=\small\sffamily,
    title={#3},
    width=(\linewidth-4mm)/2,
    height=6cm,
    colbacktitle={black},
    watermark zoom=1.0,
    watermark graphics={#2}
  ]{}%
}

% Images côte-à-côte
\noindent
\mygraphics{images/chapter-1/android-studio-1.png}{Première étape : Choisir Empty Activity}%
\hfill
\mygraphics{images/chapter-1/android-studio-2.png}{Deuxième étape : Nommer le projet}

\vspace{1em}



    {\footnotesize
\textcolor{darkgray}{\textbf{Troixième étape : } Ouvrir  \texttt{MainActivity.kt} pour voir le code généré par défaut.} 
    }


\subsection*{Explication du code par défaut}


{\footnotesize
\textcolor{darkgray}{\texttt{\textcolor{orange}{class MainActivity : ComponentActivity()}} Il s'agit de la classe principale de l'application Android.
Pour la class \texttt{\textcolor{orange}{ComponentActivity}} : Une classe de base fournie par Android pour héberger des composants Jetpack Compose.
}
}

\vspace{0.7em}

{\footnotesize
\textcolor{darkgray}{
\texttt{\textcolor{orange}{override fun onCreate(savedInstanceState: Bundle?)}} Fonction appelée automatiquement lors de la création de l'activité. C'est ici que l'on initialise l'écran grâce à la fonction \texttt{setContent\{ ... \}}.
}
}

\vspace{0.7em}

{\footnotesize
\textcolor{darkgray}{
\texttt{\textcolor{orange}{setContent\{ ... \}}} Remplace \texttt{\textcolor{orange}{setContentView()}} utilisé dans les interfaces XML. Permet de définir l'interface utilisateur directement en Compose, c'est-à-dire avec du code déclaratif en Kotlin.
}
}




\vspace{0.7em}

{\footnotesize
\textcolor{darkgray}{
\texttt{\textcolor{orange}{GestionDesTachesTheme\{ ... \}}} Thème visuel généré automatiquement par Android Studio. Il définit les couleurs, les typographies et les formes de l'application.
}
}


\vspace{0.7em}

{\footnotesize
\textcolor{darkgray}{
\texttt{\textcolor{orange}{Scaffold(...)}}
Structure de base de l'écran (layout).
Elle gère les barres d'application, le contenu principal et le bouton d'action flottant (FAB).
}
}

\vspace{0.7em}

{\footnotesize
\textcolor{darkgray}{
\texttt{\textcolor{orange}{Greeting(name: String)}}
Fonction annotée avec \texttt{@Composable}, ce qui signifie qu'elle produit du contenu visuel.
Elle affiche le texte \texttt{Hello Android!} .
}
}

\vspace{0.7em}

{\footnotesize
\textcolor{darkgray}{
\texttt{\textcolor{orange}{@Preview(showBackground = true)}}
Annotation permettant la prévisualisation du composant dans Android Studio, sans exécuter l'application.
La fonction \texttt{GreetingPreview()} affiche le message \texttt{Hello Android!}  sur fond clair.
}
}



\vspace{1em}

% --- Code Kotlin principal ---
\begin{ktlst}
class MainActivity : ComponentActivity() {
    override fun onCreate(savedInstanceState: Bundle?) {
        super.onCreate(savedInstanceState)
        setContent {
            GestiondesTachesTheme {
                Scaffold(modifier = Modifier.fillMaxSize()) { innerPadding ->
                    Greeting(
                        name = "Android",
                        modifier = Modifier.padding(innerPadding)
                    )
                }
            }
        }
    }
}

@Composable
fun Greeting(name: String, modifier: Modifier = Modifier) {
    Text(
        text = "Hello \$name!",
        modifier = modifier
    )
}

@Preview(showBackground = true)
@Composable
fun GreetingPreview() {
    GestiondesTachesTheme {
        Greeting("Android")
    }
}
\end{ktlst}