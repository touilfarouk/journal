% Section avec titre coloré et cliquable vers le sommaire
\section[\texorpdfstring{Créer notre premier projet Compose}{Créer notre premier projet Compose}]%
{\hyperref[sec:sommaire]{\textcolor{orange}{Créer notre premier projet Compose}}}%
\label{sec:chapter-1}

% Texte descriptif des étapes (pas de lignes vides dans \textcolor)
{\footnotesize
\textcolor{darkgray}{\textbf{Première étape :} Lors de la création d’un nouveau projet sous Android Studio,
il faut commencer par choisir \textit{Empty Activity}, ce qui permet de démarrer
avec une structure minimale et totalement personnalisable.}
}

\vspace{0.5em}

{\footnotesize
\textcolor{darkgray}{\textbf{Deuxième étape :} Ensuite, il est nécessaire de donner un nom à votre projet,
définir le langage (Kotlin de préférence) et choisir l’emplacement de sauvegarde.}
}

% Custom command: \mygraphics[<options>]{<image-path>}{<title>}
\newcommand{\mygraphics}[3][]{%
  \tcbox[
    enhanced,
    capture=minipage,
    boxsep=0pt,top=0pt,bottom=0pt,left=0pt,right=0pt,
    boxrule=0.4pt,
    drop fuzzy shadow,
    nobeforeafter,
    colback=black!75!white,
    toptitle=2pt,bottomtitle=2pt,
    center title,
    fonttitle=\small\sffamily,
    title={#3},
    width=(\linewidth-4mm)/2,
    height=6cm,
    colbacktitle={black},
    watermark zoom=1.0,
    watermark graphics={#2}
  ]{}%
}

% Images côte-à-côte
\noindent
\mygraphics{images/android-studio-1.png}{Première étape : Choisir Empty Activity}%
\hfill
\mygraphics{images/android-studio-2.png}{Deuxième étape : Nommer le projet}

\vspace{1em}
