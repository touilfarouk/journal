\section{\hyperref[sec:sommaire]{\textcolor{orange}{Introduction}}}\label{sec:intro}

\noindent
\noindent
\begin{minipage}[t]{0.48\textwidth}
\section*{\footnotesize \textcolor{orange}{Pourquoi Compose change tout}\label{sec:intro}}

{\footnotesize \textcolor{gray}{
Bonjour et bienvenue dans ce livre consacré à Jetpack Compose.  
Si tu tiens ce livre entre tes mains, c’est probablement parce que tu sens que quelque chose a changé dans le monde du développement Android.  
Et tu as raison : avec l’arrivée de Jetpack Compose en version stable, nous entrons dans une nouvelle ère.
}}

 

\vspace{2em} % espace vertical à l'intérieur du bloc gauche

\section*{\footnotesize \textcolor{orange}{Fin de l’ère XML}\label{sec:intro}}
{\footnotesize \textcolor{gray}{
Finis les fichiers XML interminables, finis les \texttt{findViewById()} qui traînent dans tous les coins.  
Avec Compose, on écrit l’interface directement en Kotlin.  
On pense en déclaratif.  
On code plus vite.  
On code plus propre.  
Et, soyons honnêtes : on s’amuse beaucoup plus.  
}}
\end{minipage}
\hfill
\begin{minipage}[t]{0.48\textwidth}
\section*{\footnotesize \textcolor{orange}{Création simplifiée d’interfaces}\label{sec:intro}}
{\footnotesize \textcolor{gray}{
Tu veux créer un bouton personnalisé ?  
Un champ de saisie un peu spécial ?  
Ou carrément une interface complète qui réagit en temps réel ?  
Rien de plus simple : une fonction, une annotation \texttt{@Composable}, et Compose s’occupe du reste.  
}}
\vspace{2em} % espace vertical à l'intérieur du bloc droit

\section*{\footnotesize \textcolor{orange}{Un projet concret ensemble}\label{sec:intro}}
{\footnotesize \textcolor{gray}{
Dans ce livre, on ne va pas se contenter de survoler Compose.  
On va construire un vrai projet ensemble : une application de type Gestion de tâches.  
Un seul \texttt{Activity}, zéro \texttt{Fragment}. Oui, zéro.  
Et pourtant, une application complète, fonctionnelle, et moderne.  
}}
\end{minipage}
\vspace{2em} % espace vertical à l'intérieur du bloc gauche
\vspace{2em} % espace vertical à l'intérieur du bloc gauche
\begin{tcolorbox}[colback=orange!5!white, colframe=orange!80!black, 
title=\textbf{Voici ce qui t’attend}, sharp corners, boxrule=0.6pt]

\begin{itemize}
    \item[\footnotesize \textcolor{orange}{{--}}] \footnotesize \textcolor{orange}{{Les bases de Jetpack Compose} : son cycle de vie, la recomposition, et le rôle magique du compilateur.}
    \item[\footnotesize \textcolor{orange}{{--}}] \footnotesize \textcolor{orange}{{Room Database} : stocker nos tâches localement, avec même quelques requêtes SQL personnalisées.}
    \item[\footnotesize \textcolor{orange}{{--}}] \footnotesize \textcolor{orange}{{Navigation Compose} : passer d’un écran à l’autre sans prise de tête.}
    \item[\footnotesize \textcolor{orange}{{--}}] \footnotesize \textcolor{orange}{{ViewModel} : garder une logique propre et réutilisable.}
    \item[\footnotesize \textcolor{orange}{{--}}] \footnotesize \textcolor{orange}{{DataStore} : enregistrer de petites préférences utilisateurs.}
    \item[\footnotesize \textcolor{orange}{{--}}] \footnotesize \textcolor{orange}{{Dagger Hilt} : injecter nos dépendances comme des pros, sans se battre avec des \texttt{ViewModelFactory}.}
\end{itemize}

\vspace{0.5em}
{\footnotesize \textcolor{orange}{Et ce n’est pas tout :}}

\begin{itemize}
     \item[\footnotesize \textcolor{orange}{{--}}]   \footnotesize \textcolor{orange}{{ Un splash screen animé}.}
     \item[\footnotesize \textcolor{orange}{{--}}]   \footnotesize \textcolor{orange}{{ Une fonction Swipe-to-delete} ( parce que supprimer en glissant, c’est tellement satisfaisant).}
     \item[\footnotesize \textcolor{orange}{{--}}]  \footnotesize \textcolor{orange}{{  Un système Undo} pour récupérer une tâche supprimée par erreur.}
     \item[\footnotesize \textcolor{orange}{{--}}]   \footnotesize \textcolor{orange}{{Des animations fluides} pour rendre l’expérience encore plus agréable.}
     \item[\footnotesize \textcolor{orange}{{--}}]   \footnotesize \textcolor{orange}{{Le mode sombre}, activé dès le départ, parce que nos yeux le méritent.}
\end{itemize}

\end{tcolorbox}

\newpage
\noindent 
{\footnotesize \textcolor{darkgray}{Ce livre, tel un compagnon de route, sera régulièrement enrichi et mis à jour.}}
\\
{\footnotesize \textcolor{gray}{Alors… qu’attendez-vous ?}}
\footnotesize \textcolor{orange}{{Tournez la page, et entrons ensemble dans cette nouvelle ère du développement Android.}}