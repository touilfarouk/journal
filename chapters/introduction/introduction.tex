\section{\hyperref[sec:summary]{\textcolor{orange}{Introduction}}}\label{sec:intro}
\noindent
\noindent
\begin{minipage}[t]{0.48\textwidth}
\section*{Introduction à Jetpack Compose}

Bonjour et bienvenue dans ce livre consacré à Jetpack Compose.  
Si tu tiens ce livre entre tes mains, c’est probablement parce que tu sens que quelque chose a changé dans le monde du développement Android.  
Et tu as raison : avec l’arrivée de Jetpack Compose en version stable, nous entrons dans une nouvelle ère.  

\vspace{2em} % espace vertical à l'intérieur du bloc gauche

\section*{Fin de l’ère XML}

Finis les fichiers XML interminables, finis les \texttt{findViewById()} qui traînent dans tous les coins.  
Avec Compose, on écrit l’interface directement en Kotlin.  
On pense en déclaratif.  
On code plus vite.  
On code plus propre.  
Et, soyons honnêtes : on s’amuse beaucoup plus.  
\end{minipage}
\hfill
\begin{minipage}[t]{0.48\textwidth}
\section*{Création simplifiée d’interfaces}

Tu veux créer un bouton personnalisé ?  
Un champ de saisie un peu spécial ?  
Ou carrément une interface complète qui réagit en temps réel ?  
Rien de plus simple : une fonction, une annotation \texttt{@Composable}, et Compose s’occupe du reste.  

\vspace{2em} % espace vertical à l'intérieur du bloc droit

\section*{Un projet concret ensemble}

Dans ce livre, on ne va pas se contenter de survoler Compose.  
On va construire un vrai projet ensemble : une application de type Gestion de tâches.  
Un seul \texttt{Activity}, zéro \texttt{Fragment}. Oui, zéro.  
Et pourtant, une application complète, fonctionnelle, et moderne.  
\end{minipage}
\vspace{2em} % espace vertical à l'intérieur du bloc gauche
\vspace{2em} % espace vertical à l'intérieur du bloc gauche
\begin{tcolorbox}[colback=orange!5!white, colframe=orange!80!black, 
title=\textbf{Voici ce qui t’attend}, sharp corners, boxrule=0.6pt]

\begin{itemize}
    \item[\textbf{--}] \textbf{Les bases de Jetpack Compose} : son cycle de vie, la recomposition, et le rôle magique du compilateur.
    \item[\textbf{--}] \textbf{Room Database} : stocker nos tâches localement, avec même quelques requêtes SQL personnalisées.
    \item[\textbf{--}] \textbf{Navigation Compose} : passer d’un écran à l’autre sans prise de tête.
    \item[\textbf{--}] \textbf{ViewModel} : garder une logique propre et réutilisable.
    \item[\textbf{--}] \textbf{DataStore} : enregistrer de petites préférences utilisateurs.
    \item[\textbf{--}] \textbf{Dagger Hilt} : injecter nos dépendances comme des pros, sans se battre avec des \texttt{ViewModelFactory}.
\end{itemize}

\vspace{0.5em}
\textbf{Et ce n’est pas tout :}

\begin{itemize}
    \item[$\star$] Un \textbf{splash screen animé}.
    \item[$\star$] Une fonction \textbf{Swipe-to-delete} (parce que supprimer en glissant, c’est tellement satisfaisant).
    \item[$\star$] Un système \textbf{Undo} pour récupérer une tâche supprimée par erreur.
    \item[$\star$] Des \textbf{animations fluides} pour rendre l’expérience encore plus agréable.
    \item[$\star$] Le \textbf{mode sombre}, activé dès le départ, parce que nos yeux le méritent.
\end{itemize}

\end{tcolorbox}

\vspace{1em}

\noindent Mais ce livre n’est pas seulement une suite d’exemples.  
Il est pensé comme un \textbf{compagnon de route}.  

Tu vas apprendre à structurer tes projets, à écrire un code plus propre, à comprendre ce qui se passe sous le capot.  

Bref : à devenir un \textbf{développeur Android moderne}.  

\vspace{0.5em}
\begin{center}
\textit{Alors, es-tu prêt à tourner la page et à te lancer dans Compose ?}\\
Parce que l’avenir du développement Android commence maintenant.\\
\textbf{Et il commence avec toi.}
\end{center}

