\section{\hyperref[sec:sommaire]{\textcolor{orange}{Introduction}}}\label{sec:intro}

\noindent
\noindent
\begin{minipage}[t]{0.48\textwidth}
\section*{\footnotesize \textcolor{orange}{Pourquoi Compose change tout}}

{\footnotesize \textcolor{gray}{
Bonjour et bienvenue dans ce livre consacré à Jetpack Compose.  
Si tu tiens ce livre entre tes mains, c’est probablement parce que tu sens que quelque chose a changé dans le monde du développement Android.  
Et tu as raison : avec l’arrivée de Jetpack Compose en version stable, nous entrons dans une nouvelle ère.
}}

 

\vspace{2em} % espace vertical à l'intérieur du bloc gauche

\section*{\footnotesize \textcolor{orange}{Fin de l’ère XML}}
{\footnotesize \textcolor{gray}{
Finis les fichiers XML interminables, finis les \texttt{findViewById()} qui traînent dans tous les coins.  
Avec Compose, on écrit l’interface directement en Kotlin.  
On pense en déclaratif.  
On code plus vite.  
On code plus propre.  
Et, soyons honnêtes : on s’amuse beaucoup plus.  
}}
\end{minipage}
\hfill
\begin{minipage}[t]{0.48\textwidth}
\section*{Création simplifiée d’interfaces}

Tu veux créer un bouton personnalisé ?  
Un champ de saisie un peu spécial ?  
Ou carrément une interface complète qui réagit en temps réel ?  
Rien de plus simple : une fonction, une annotation \texttt{@Composable}, et Compose s’occupe du reste.  

\vspace{2em} % espace vertical à l'intérieur du bloc droit

\section*{Un projet concret ensemble}

Dans ce livre, on ne va pas se contenter de survoler Compose.  
On va construire un vrai projet ensemble : une application de type Gestion de tâches.  
Un seul \texttt{Activity}, zéro \texttt{Fragment}. Oui, zéro.  
Et pourtant, une application complète, fonctionnelle, et moderne.  
\end{minipage}
\vspace{2em} % espace vertical à l'intérieur du bloc gauche
\vspace{2em} % espace vertical à l'intérieur du bloc gauche
\begin{tcolorbox}[colback=orange!5!white, colframe=orange!80!black, 
title=\textbf{Voici ce qui t’attend}, sharp corners, boxrule=0.6pt]

\begin{itemize}
    \item[\textbf{--}] \textbf{Les bases de Jetpack Compose} : son cycle de vie, la recomposition, et le rôle magique du compilateur.
    \item[\textbf{--}] \textbf{Room Database} : stocker nos tâches localement, avec même quelques requêtes SQL personnalisées.
    \item[\textbf{--}] \textbf{Navigation Compose} : passer d’un écran à l’autre sans prise de tête.
    \item[\textbf{--}] \textbf{ViewModel} : garder une logique propre et réutilisable.
    \item[\textbf{--}] \textbf{DataStore} : enregistrer de petites préférences utilisateurs.
    \item[\textbf{--}] \textbf{Dagger Hilt} : injecter nos dépendances comme des pros, sans se battre avec des \texttt{ViewModelFactory}.
\end{itemize}

\vspace{0.5em}
\textbf{Et ce n’est pas tout :}

\begin{itemize}
    \item[$\star$] Un \textbf{splash screen animé}.
    \item[$\star$] Une fonction \textbf{Swipe-to-delete} (parce que supprimer en glissant, c’est tellement satisfaisant).
    \item[$\star$] Un système \textbf{Undo} pour récupérer une tâche supprimée par erreur.
    \item[$\star$] Des \textbf{animations fluides} pour rendre l’expérience encore plus agréable.
    \item[$\star$] Le \textbf{mode sombre}, activé dès le départ, parce que nos yeux le méritent.
\end{itemize}

\end{tcolorbox}

\newpage
\noindent Dans ce livre, nous allons manipuler certains des composants d’architecture Android les plus essentiels.
Par exemple, nous utiliserons Room Database pour enregistrer et lire nos données localement. Et, bien sûr, nous irons plus loin qu’un simple CRUD : nous écrirons également quelques requêtes SQL personnalisées pour tirer pleinement parti de notre base de données.

Ensuite, nous aborderons Compose Navigation, afin de passer d’un écran à l’autre sans la complexité habituelle.
Puis, au fil du chemin, nous intégrerons un ViewModel, véritable cœur logique de notre application, qui nous permettra de centraliser et de structurer le code métier.
À cela viendra s’ajouter le Preference DataStore, parfait pour sauvegarder de simples couples clé–valeur de manière moderne et sécurisée.

Mais ce n’est pas tout. Vous découvrirez également une bibliothèque très populaire d’injection de dépendances : Dagger Hilt. Grâce à elle, plus besoin de créer manuellement des ViewModelFactory. Nous injecterons nos dépendances proprement, et surtout efficacement.

Et parce qu’une application moderne ne se limite pas à sa logique interne, nous soignerons aussi son apparence.
Notre application prendra en charge le mode sombre dès le départ. Vous verrez combien il est simple, avec Jetpack Compose, d’offrir plusieurs thèmes à vos utilisateurs.
Nous créerons ensemble des composants UI personnalisés, tout en exploitant les composants Material Design les plus importants.
Nous découvrirons aussi le Scaffold, un composable puissant qui utilise l’API des “slots” pour organiser vos interfaces de manière cohérente et fidèle aux recommandations de Material Design.

Et ce n’est pas fini. Notre application  Gestion de tâches inclura :

un splash screen animé,

une fonctionnalité Swipe-to-delete élégante et intuitive,

ainsi qu’une option UndoDeletedItem qui permettra de restaurer une tâche supprimée en un simple clic grâce à une snackbar.

Tout cela donnera vie à une application non seulement fonctionnelle, mais aussi agréable à utiliser.

Enfin, au-delà des aspects techniques, vous apprendrez bien plus qu’un enchaînement d’API et de composants : vous découvrirez une nouvelle façon de penser le développement Android.
Ce livre, tel un compagnon de route, sera régulièrement enrichi et mis à jour.
Alors… qu’attendez-vous ?
\textbf{Tournez la page, et entrons ensemble dans cette nouvelle ère du développement Android.}